\documentclass[12pt, letterpaper, titlepage]{article}
\usepackage[left=3.5cm, right=2.5cm, top=2.5cm, bottom=2.5cm]{geometry}
\usepackage[MeX]{polski}
\usepackage[utf8]{inputenc}
\usepackage{graphicx}
\usepackage{multirow}
\usepackage{caption}
\usepackage{xcolor}
\usepackage{enumerate}
\usepackage{amsmath} %pakiet matematyczny
\usepackage{amssymb} %pakiet dodatkowych symboli
\title{Tabele w TeXu}
\author{Michał Pączkowski}
\date{Listopad 2022}

\begin{document}
\maketitle
\section{Zadanie}
\begin{table}[h]
\centering\caption{Przykładowy system decyzyjny (U,A,d), modelujący problem diagnozy medycznej, której efektem jest decyzja o wykonaniu lub nie wykonaniu operacji wycięcia wyrostka robaczkowego,\\ U = $\{u1, u2,...,u10\}$, A = $\{a1, a2\}$, d $\in$ D=$ \{TAK, NIE\}$}
\begin{tabular}{ c | c c c }
\hline
\hline
Pacjent & Ból brzucha & Temperatura ciała & Operacja\\
\hline
u1 & Mocny & Wysoka & Tak \\
u2 & Średni & Wysoka & Tak \\
u3 & Mocny & Średnia & Tak \\
u4 & Mocny & Niska & Tak \\
u5 & Średni & Średnia & Tak \\
u6 & Średni & Średnia & Nie \\
u7 & Mały & Wysoka & Nie \\
u8 & Mały & Niska & Nie \\
u9 & Mocny & Niska & Nie \\
u10 & Mały & Średnia & Nie \\
\hline
\hline
\end{tabular}
\end{table}

\section{Zadanie}
\textit{Bramki logiczne}
\begin{table}[h]
\centering
\caption*{NOT}
\begin{tabular}{ c | c }
\hline
\hline
Wejście & Wyjście\\
\hline
0 & 1\\
1 & 0\\
\hline
\hline
\end{tabular}
\end{table}

\begin{table}[h]
\centering
\caption*{AND}
\begin{tabular}{ c  c | c }
\hline
\hline
A & B & Wyjście\\
\hline
0 & 0 & 0\\
0 & 1 & 0\\
1 & 0 & 0\\
1 & 1 & 1\\
\hline
\hline
\end{tabular}
\end{table}

\begin{table}[h]
\centering
\caption*{OR}
\begin{tabular}{ c  c | c }
\hline
\hline
A & B & Wyjście\\
\hline
0 & 0 & 0\\
0 & 1 & 1\\
1 & 0 & 1\\
1 & 1 & 1\\
\hline
\hline
\end{tabular}
\end{table}

\begin{table}[h]
\centering
\caption*{NAND}
\begin{tabular}{ c  c | c }
\hline
\hline
A & B & Wyjście\\
\hline
0 & 0 & 1\\
0 & 1 & 1\\
1 & 0 & 1\\
1 & 1 & 0\\
\hline
\hline
\end{tabular}
\end{table}

\begin{table}[h]
\centering
\caption*{NOR}
\begin{tabular}{ c  c | c }
\hline
\hline
A & B & Wyjście\\
\hline
0 & 0 & 1\\
0 & 1 & 0\\
1 & 0 & 0\\
1 & 1 & 0\\
\hline
\hline
\end{tabular}
\end{table}

\begin{table}[h]
\centering
\caption*{XOR}
\begin{tabular}{ c  c | c }
\hline
\hline
A & B & Wyjście\\
\hline
0 & 0 & 0\\
0 & 1 & 1\\
1 & 0 & 1\\
1 & 1 & 0\\
\hline
\hline
\end{tabular}
\end{table}

\end{document}
\documentclass[a4paper]{article}
\usepackage[left=3.5cm, right=2.5cm, top=2.5cm, bottom=2.5cm]{geometry}
\usepackage[MeX]{polski}
\usepackage[utf8]{inputenc}
\usepackage{graphicx}
\usepackage{enumerate}
\usepackage{amsmath} %pakiet matematyczny
\usepackage{amssymb} %pakiet dodatkowych symboli
\title{Pierwszy dokument LaTeX}
\author{Michał Pączkowski}
\date{15 Październik 2022}
\begin{document}
\maketitle
\newpage
\section{Kim jestem?}
\textit{Jestem młodym człowiekiem, któremu rodzice dali imię Michał. Mam aktualnie \underline{20 lat} i ukończyłem \underline{technikum informatyczne} wraz ze wszystkimi kwalifikacjami.\newline
\textbf{Moja osoba charakteryzuje się takimi cechami jak:}
\begin{enumerate}[I.]
\item pracowitość %punktowa tabela
\item sumienność %punktowa tabela
\item umiejętność pracy w zespole %punktowa tabela
\item punktualność %punktowa tabela
\end{enumerate}
\subsection{Dlaczego akurat studia informatyczne?}
Trafiłem tutaj przypadkiem, matura została dobrze napisana, więc trzeba było iść szkolić się dalej. Z racji, że nie wiedziałem co chcę w życiu robić, poszedłem dalej w kierunku technologii :)\newline
To wszystko jest bardzo proste:
\begin{itemize}
\item {Chciałbym zaczerpnąć nową wiedzę.}
\item {Chciałbym poznać nowych ciekawych ludzi.}
\item {Chciałbym zdobyć odpowiednie papiery inżynierskie.}
\item {Chciałbym zaczerpnąć czegoś nowego.}
\end{itemize}
\subsubsection{Czym zajmuje się na co dzień?}
Na co dzień pracuje, łapie prace dorywcze. Próbuje swoich sił w internecie, no cóż. Każdy orze jak może.}
\newpage
\section{NEWS DNIA}

\paragraph{\LARGE{Warszawa: Uczeń przyszedł do szkoły z pistoletem}}
\subparagraph{\\\Large{Do jednej ze szkół na warszawskim Mokotowie uczeń przyniósł broń. Szkoła zawiadomiła straż miejską.\\\\}}
\large{Około godziny 11 w czwartek funkcjonariusze wspólny patrolu szkolnego straży miejskiej i policji zostali wezwani do placówki przy ul. Różanej. Poinformowano ich o uczniu, który straszył pistoletem kolegów.
W budynku czekała na patrol nauczycielka, która przekazała, że podczas przerwy usłyszała krzyk dzieci na pierwszym piętrze szkoły. Gdy tam pobiegła usłyszała od uczniów, że jeden z ich kolegów przyniósł do szkoły pistolet.}
\subparagraph{\\\Large {Odebrała pistolet uczniowi\\\\}}
\large{Nie zważając na swoje bezpieczeństwo kobieta postanowiła interweniować. Nauczycielka odnalazła ucznia i zabrała mu pistolet, który przekazała dyrekcji placówki.
Okazało się na szczęście, że pistolet to nie była broń palna, a niezłej jakości replika na plastikowe kulki wystrzeliwane za pomocą sprężonego dwutlenku węgla.
Do placówki wezwano rodziców chłopca, którym przekazano ucznia.}
\end{document}


